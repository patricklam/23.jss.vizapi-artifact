\section{Potential Applications}
\label{sec:potential-applications}
While the main contribution of this paper is to share our experiences
about writing program analysis and transformation tools, we also
briefly outline some potential applications of VizAPI itself.

In our original paper, we discussed how 1) library developers could
use VizAPI to guide their API maintenance decisions---based on actual
usage, which APIs were best prunable or modifiable, and where good
refactoring candidates might exist, thus helping library users'
experiences; and 2) client developers could use VizAPI to guide
decisions about when to upgrade their dependencies while avoiding
upgrade pains.

An extension of VizAPI could help with the problem of library selection. In
Section~\ref{sec:design-decisions} we discussed considerations for library
selection for VizAPI-like tools. An extension to VizAPI could also help
general developers pick which libraries would be helpful for their
needs---''projects like you used libraries X and Y''. In particular,
VizAPI can help surface the specific packages that similar projects used,
helping developers understand the functionalities of candidate libraries.

Our static and dynamic analyses record versions of different libraries used. 
Examining the data to observe how often clients upgrade library versions, 
whether clients use multiple versions of the same library is possibly interesting research for library developers.
\todo{library version usage: does this make sense?}

Another area of potential applications would be security; we already
identify {\texttt ServiceLoader} bypasses, and could also identify
other suspicious patterns. VizAPI would be particularly good for
identifying cases where vulnerabilities from dependencies do and do
not propagate; lack of an edge from a vulnerability-containing source
to a sink in a client would be a hint that the client would not be
vulnerable to the vulnerability. Unexpected edges could also indicate
abnormal behaviours that should be further investigated.
VizAPI could detect violations of
security policies, though it is admittedly not the best tool for that.

More fundamentally, we believe that the way VizAPI combines static and
dynamic information could be useful in understanding the limitations
of static and dynamic approaches in practice. In
Section~\ref{sec:design-decisions}, we talked about conceptual
limitations of both approaches. Sui et
al~\cite{sui20:_recal_static_call_graph_const_pract} have investigated
the recall of static call graph construction in practice, but further
work in this area would help in terms of having confidence in the
analysis results that we compute and rely on.

%% Potential Applications:
%% Security: 
%% Identify illegal accesses and abnormal usage
%% Identify propagated vulnerabilities from dependencies
%% Developer experience: analyze dataset for improving experience of library users
%% Help library developers identify which features to focus efforts on
%% Predict good choices of libraries based on client use case
%% Analyze library version usage: how often clients upgrade their library versions, do they use 
%% How does static and dynamic usage differ?
%% (other ML applications of the data??)
