\documentclass{letter}
%use letter format
\setlength{\paperheight}{11in}
\setlength{\paperwidth}{8.5in}
\setlength{\pdfpageheight}{11in}
\setlength{\pdfpagewidth}{8.5in}

\usepackage{cite}
\usepackage{letterbib}

\setlength{\topmargin}{-.5in}
\setlength{\headheight}{0in}
\setlength{\textheight}{8.5in}

\signature{Sruthi Venkatanarayan, Craig Anslow, and Patrick Lam}

\begin{document}
\begin{letter}{}
\opening{Dear Drs. Papoutsoglou and Treude,}

We are submitting the attached paper, ``Design and implementation of the VizAPI tool for visualizing static and dynamic analysis of Java software,'' for publication in the JSS Special Issue on Open Science in Software Engineering Research. The main contributions of this paper are:
\begin{itemize}
\item it describes the design and implementation of the artifacts behind the VizAPI tool (Section 3);
\item it discusses design decisions behind VizAPI---particularly the relevant tools in the visualization and Java static and dynamic analysis spaces, and how to choose an appropriate tool (Section 4).
We particularly hope that this section will be useful to other researchers;
\item it describes the benchmark suite that we collected to evaluate VizAPI.
\end{itemize}

We have made our artifacts available both on zenodo (for archival
purposes) and Github (for ease of extensibility). The paper contains
all relevant links.

We have attached the VISSOFT NIER paper ``VizAPI: Visualizing
Interactions between Java Libraries and Clients'' to this submission
for reference. Conceptually, the VISSOFT paper described the tool from
a user's perspective, while the present paper describes the tool's
implementation. The most significant repeated content is in Section
3.1 and Figure 3, giving examples of VizAPI usage. These account for
two pages (out of 12 pages in this submission). There are also a few
more paragraphs across Section 3 with repeated content. Some of the
material here comes from the firast author's MMath thesis, which is
not a peer-reviewed publication. We are confident that this paper
contains well over 70\% original content (8.4 pages).

\closing{Thank you for your consideration.}

%\small
%\bibliographystyle{plain}
%\bibliography{refs}

\end{letter} 

\end{document}
