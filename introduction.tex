\section{Introduction}
\label{sec:introduction}

In previous work, we have presented the VizAPI tool~\cite{venkatanarayanan22:_vizap,venkatanarayanan22:_study_lever_api_usage_patter}, which creates visualization overviews showing API usages---mostly from clients to libraries, but also between libraries (including transitive dependencies). The goal of VizAPI is to provide a heuristic for developers considering the impacts of changes to libraries. VizAPI incorporates information from static and dynamic analyses and uses the d3 visualization toolkit to present this information to developers. We have made VizAPI publicly available\footnote{\url{https://github.com/SruthiVenkat/api-visualization-tool}}.

This work aims to explain the VizAPI artifact and to share some of our knowledge in developing software analysis and transformation tools. We describe the design and implementation of VizAPI (Section~\ref{sec:implementation}). We then continue by discussing design alternatives that we could have taken, and especially other libraries that we could have used (Section~\ref{sec:design-decisions}). Dataset construction is challenging, and we explain the dataset that we used to evaluate VizAPI (Section~\ref{sec:benchmark}). We also outline some potential uses of VizAPI's techniques in future research (Section~\ref{sec:potential-applications}).



